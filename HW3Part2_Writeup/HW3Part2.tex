\documentclass{article}

\usepackage{fancyhdr}
\usepackage{extramarks}
\usepackage{amsmath}
\usepackage{amsthm}
\usepackage{amsfonts}
\usepackage{tikz}
\usepackage[plain]{algorithm}
\usepackage{algpseudocode}

\usetikzlibrary{automata,positioning}

%
% Basic Document Settings
%

\topmargin=-0.45in
\evensidemargin=0in
\oddsidemargin=0in
\textwidth=6.5in
\textheight=9.0in
\headsep=0.25in

\linespread{1.1}

\pagestyle{fancy}
\rhead{\firstxmark}
\lfoot{\lastxmark}
\cfoot{\thepage}

\renewcommand\headrulewidth{0.4pt}
\renewcommand\footrulewidth{0.4pt}

\setlength\parindent{0pt}

%
% Create Problem Sections
%

\newcommand{\enterProblemHeader}[1]{
    \nobreak\extramarks{}{Problem \arabic{#1} continued on next page\ldots}\nobreak{}
    \nobreak\extramarks{Problem \arabic{#1} (continued)}{Problem \arabic{#1} continued on next page\ldots}\nobreak{}
}

\newcommand{\exitProblemHeader}[1]{
    \nobreak\extramarks{Problem \arabic{#1} (continued)}{Problem \arabic{#1} continued on next page\ldots}\nobreak{}
    \stepcounter{#1}
    \nobreak\extramarks{Problem \arabic{#1}}{}\nobreak{}
}

\DeclareMathOperator*{\argmax}{\arg\!\max}

\setcounter{secnumdepth}{0}
\newcounter{partCounter}
\newcounter{homeworkProblemCounter}
\setcounter{homeworkProblemCounter}{1}
\nobreak\extramarks{Problem \arabic{homeworkProblemCounter}}{}\nobreak{}

%
% Homework Problem Environment
%
% This environment takes an optional argument. When given, it will adjust the
% problem counter. This is useful for when the problems given for your
% assignment aren't sequential. See the last 3 problems of this template for an
% example.
%
\newenvironment{homeworkProblem}[1][-1]{
    \ifnum#1>0
        \setcounter{homeworkProblemCounter}{#1}
    \fi
    \section{Problem \arabic{homeworkProblemCounter}}
    \setcounter{partCounter}{1}
    \enterProblemHeader{homeworkProblemCounter}
}{
    \exitProblemHeader{homeworkProblemCounter}
}



%
% Homework Details
%   - Title
%   - Due date
%   - Class
%   - Section/Time
%   - Instructor
%   - Author
%

\newcommand{\hmwkTitle}{Homework\ \#3 Part 2}
\newcommand{\hmwkName}{Probabilistic Reasoning}
\newcommand{\hmwkDate}{May 2, 2023}
\newcommand{\hmwkClass}{CS 440: Introduction to Artificial Intelligence}
\newcommand{\hmwkClassInstructor}{Professor Abdelsam Boularias}


%
% Title Page
%

\title{
    \vspace{2in}
    \textmd{\textbf{\hmwkClass}}\\
    \textmd{\hmwkTitle\ \hmwkName}\\
    \vspace{0.1in}\small{\hmwkDate}\\
    \vspace{0.1in}\large{\textit{\hmwkClassInstructor}}
    \vspace{3in}
}

\author{
    \textbf{Jay Patwardhan} {208001851}\\ 
    \textbf{Alan Wu} {208000574}\\ 
    \textbf{Neel Shejwalkar} {207004853}
}
\date{}

\renewcommand{\part}[1]{\textbf{\large Part \Alph{partCounter}}\stepcounter{partCounter}\\}

%
% Various Helper Commands
%

% Useful for algorithms
\newcommand{\alg}[1]{\textsc{\bfseries \footnotesize #1}}

% For derivatives
\newcommand{\deriv}[1]{\frac{\mathrm{d}}{\mathrm{d}x} (#1)}

% For partial derivatives
\newcommand{\pderiv}[2]{\frac{\partial}{\partial #1} (#2)}

% Integral dx
\newcommand{\dx}{\mathrm{d}x}

% Alias for the Solution section header
\newcommand{\bruh}{\textbf{\large Solution}}
\newtheorem*{solution*}{Solution}
\newenvironment{solution}{\begin{solution*}}{{\finishline} \end{solution*}}


% Probability commands: Expectation, Variance, Covariance, Bias
\newcommand{\E}{\mathrm{E}}
\newcommand{\Var}{\mathrm{Var}}
\newcommand{\Cov}{\mathrm{Cov}}
\newcommand{\Bias}{\mathrm{Bias}}

\begin{document}

\maketitle

\pagebreak

%Part 1%
\begin{homeworkProblem}
   Before we begin to tackle these problems, let's define the transition and observation matrices to make the calculations easier.
   \[
      T =
      \begin{bmatrix}
         0.2 & 0.8 & 0   & 0   & 0   & 0   \\
         0   & 0.2 & 0.8 & 0   & 0   & 0   \\
         0   & 0   & 0.2 & 0.8 & 0   & 0   \\
         0   & 0   & 0   & 0.2 & 0.8 & 0   \\
         0   & 0   & 0   & 0   & 0.2 & 0.8 \\
         0   & 0   & 0   & 0   & 0   & 0
      \end{bmatrix}
   \]
   \[
      O_{hot} =
      \begin{bmatrix}
         1 & 0 & 0 & 0 & 0 & 0 \\
         0 & 0 & 0 & 0 & 0 & 0 \\
         0 & 0 & 0 & 0 & 0 & 0 \\
         0 & 0 & 0 & 1 & 0 & 0 \\
         0 & 0 & 0 & 0 & 0 & 0 \\
         0 & 0 & 0 & 0 & 0 & 0
      \end{bmatrix}
   \]
   \[
      O_{cold} =
      \begin{bmatrix}
         0 & 0 & 0 & 0 & 0 & 0 \\
         0 & 1 & 0 & 0 & 0 & 0 \\
         0 & 0 & 1 & 0 & 0 & 0 \\
         0 & 0 & 0 & 0 & 0 & 0 \\
         0 & 0 & 0 & 0 & 1 & 0 \\
         0 & 0 & 0 & 0 & 0 & 1
      \end{bmatrix}
   \]

   \textbf{1. Filtering} \\
   In order to find $P(X_3 | hot_1, cold_2, cold_3)$, we can use the forward model:
   \begin{align*}
      P(X_3 | hot_1, cold_2, cold_3) & = O_{hot}*T^T*P(X_2 | hot_1, cold_2)      \\
                                     & = O_{hot}*T^T*O_{cold}*T^T*P(X_1 | hot_1)
   \end{align*}
   where $P(X_1 | hot_1)$ is simply $[1,0,0,0,0,0]$ as stated in the problem (because the rover is guaranteed to be on location A on the first day). So then,
   \begin{align*}
      O_{hot}*T^T*O_{cold}*T^T*[1,0,0,0,0,0]^T = [0,0.16,0.64,0,0,0]
   \end{align*}
   which, after normalization, becomes
   \begin{align*}
      [0,0,2,0.8,0,0,0] . \\
   \end{align*}

   \textbf{2. Smoothing} \\
   To find $P(X_2 | hot_1, cold_2, cold_3)$, we can split up the calculation according into a backward model and a forward model to get the smoothed probability.
   \begin{align*}
      P(X_2 | hot_1, cold_2, cold_3) & = b_3 * f_{1:2} \\
   \end{align*}
   where
   \begin{align*}
      b_3 & = T*O_{cold}*b_4             \\
          & = T*O_{cold}*[1,1,1,1,1,1]^T \\
          & = [0.8, 1, 0.2, 0.8, 1, 0]
   \end{align*}
   and
   \begin{align*}
      f_{1:2} & = O_{cold}*T^T*P(X_1|hot_1) \\
              & = [0, 0.8, 0, 0, 0, 0]
   \end{align*}
   which makes
   \begin{align*}
      P(X_2 | hot_1, cold_2, cold_3) & = b_3 * f_{1:2}                                   \\
                                     & = [0.8, 1, 0.2, 0.8, 1, 0] * [0, 0.8, 0, 0, 0, 0] \\
                                     & = [0,0.8,0,0,0,0] .
   \end{align*}
   After normalization, this becomes
   \begin{align*}
      [0,1,0,0,0,0] .
   \end{align*}

   \textbf{4. Prediction} \\
   In order to find the distribution $P(X_4 | hot_1, cold_2, cold_3)$, we can simply split this up into:
   \begin{align*}
      P(X_4 | hot_1, cold_2, cold_3) & = T^T*P(X_3 | hot_1, cold_2, cold_3) \\
                                     & = T^T*[0,0.2,0.8,0,0,0]^T            \\
                                     & = [0,0.04,0.32,0.64,0,0] .
   \end{align*}
   We just used the normalized $P(X_3 | hot_1, cold_2, cold_3)$ so as to bypass normalizing this new distribution again. \\

   \textbf{3. Prediction} \\
   We can use the distribution from part 4 to calculate $P(hot_4 | hot_1, cold_2, cold_3)$.
   \begin{align*}
      P(hot_4 | hot_1, cold_2, cold_3) &= P(X_4 = 1 \textnormal{ or } X_4 = 4 | hot_1, cold_2, cold_3) \\
      &= 0 + 0.64 \\
      &= 0.64
   \end{align*}
\end{homeworkProblem}

\begin{homeworkProblem}
   \textbf{1. Bellman Equation} \\
   The Bellman equation for this particular MDP would be:
   \begin{align*}
      V^{\pi}(s) & = R(s, \pi(s)) + \sum_{s' \in \lbrace A, B \rbrace} T(s, \pi(s), s') V^{\pi}(s') \\
   \end{align*}

   \textbf{2. Policy Evaluation} \\
   Using the initial policy rules $\pi(A)=1$ and $\pi(B)=1$ with initial policy values $V^\pi_0(A) = V^\pi_0(A) = 0$, two iterations of policy evaluation would look like this:
   \begin{enumerate}
      \item Iteration 1:
            \begin{align*}
               A: V^\pi_1(A) & = R(A, \pi_0(A)) + [T(A, \pi_0(A), A) V^{\pi}_0(A) + T(A, \pi_0(A), B) V^{\pi}_0(B)] \\
                             & = 0 + [0+0]                                                                          \\
                             & = 0                                                                                  \\
               B: V^\pi_1(B) & = R(B, \pi_0(B)) + [T(B, \pi_0(B), A) V^{\pi}_0(A) + T(B, \pi_0(B), B) V^{\pi}_0(B)] \\
                             & = 5 + [0 + 0]                                                                        \\
                             & = 5
            \end{align*}
      \item Iteration 2:
            \begin{align*}
               A: V^\pi_2(A) & = R(A, \pi_0(A)) + [T(A, \pi_0(A), A) V^{\pi}_1(A) + T(A, \pi_0(A), B) V^{\pi}_1(B)] \\
                             & = 0 + [0+0]                                                                          \\
                             & = 0                                                                                  \\
               B: V^\pi_2(B) & = R(B, \pi_0(B)) + [T(B, \pi_0(B), A) V^{\pi}_1(A) + T(B, \pi_0(B), B) V^{\pi}_1(B)] \\
                             & = 5 + [0 + 5]                                                                        \\
                             & = 10
            \end{align*}
   \end{enumerate}
   So the updated values are $V^\pi(A) = 0$ and $V^\pi(A) = 10$. \\

   \textbf{3. Improved Policy} \\
   We can find $\pi_{new}$ using the policy improvement equation:
   \begin{align*}
      \pi_{new}(A) & = \argmax_{a \in \lbrace 1,2 \rbrace}[R(A,a)+(T(A,a,A)V_2(A)+T(A,a,B)V_2(B))] \\
                   & = 2 \\
      \pi_{new}(B) & = \argmax_{a \in \lbrace 1,2 \rbrace}[R(B,a)+(T(B,a,A)V_2(A)+T(B,a,B)V_2(B))] \\
                   & = 1 
   \end{align*}
   The work is shown below.
   \begin{itemize}
      \item $\pi_{new}(A)$
      \begin{itemize}
         \item a=1: $0 + 1*0 + 0*10 = 0$
         \item a=2: $-1 + 0.5*0 + 0.5*10 = 4$
      \end{itemize}

      \item $\pi_{new}(B)$
      \begin{itemize}
         \item a=1: $5 + 0 + 1*10 = 15$
         \item a=2: $0 + 0 + 1*10 = 10$
      \end{itemize}
   \end{itemize}

\end{homeworkProblem}

\end{document}